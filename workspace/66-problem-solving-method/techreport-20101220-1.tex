% Clever Algorithms: Problem Solving Methodology

% The Clever Algorithms Project: http://www.CleverAlgorithms.com
% (c) Copyright 2010 Jason Brownlee. Some Rights Reserved. 
% This work is licensed under a Creative Commons Attribution-Noncommercial-Share Alike 2.5 Australia License.

\documentclass[a4paper, 11pt]{article}
\usepackage{tabularx}
\usepackage{booktabs}
\usepackage{url}
\usepackage[pdftex,breaklinks=true,colorlinks=true,urlcolor=blue,linkcolor=blue,citecolor=blue,]{hyperref}
\usepackage{geometry}
\geometry{verbose,a4paper,tmargin=25mm,bmargin=25mm,lmargin=25mm,rmargin=25mm}

% Dear template user: fill these in
\newcommand{\myreporttitle}{Clever Algorithms}
\newcommand{\myreportsubtitle}{Problem Solving Methodology}
\newcommand{\myreportauthor}{Jason Brownlee}
\newcommand{\myreportemail}{jasonb@CleverAlgorithms.com}
\newcommand{\myreportproject}{The Clever Algorithms Project\\\url{http://www.CleverAlgorithms.com}}
\newcommand{\myreportdate}{20101220}
\newcommand{\myreportfulldate}{December 20, 2010}
\newcommand{\myreportversion}{1}
\newcommand{\myreportlicense}{\copyright\ Copyright 2010 Jason Brownlee. Some Rights Reserved. This work is licensed under a Creative Commons Attribution-Noncommercial-Share Alike 2.5 Australia License.}

% leave this alone, it's templated baby!
\title{{\myreporttitle}: {\myreportsubtitle}\footnote{\myreportlicense}}
\author{\myreportauthor\\{\myreportemail}\\\small\myreportproject}
\date{\myreportfulldate\\{\small{Technical Report: CA-TR-{\myreportdate}-\myreportversion}}}
\begin{document}
\maketitle

% write a summary sentence for each major section
\section*{Abstract} 
% project
The Clever Algorithms project aims to describe a large number of Artificial Intelligence algorithms in a complete, consistent, and centralized manner, to improve their general accessibility. 
% template
The project makes use of a standardized algorithm description template that uses well-defined topics that motivate the collection of specific and useful information about each algorithm described.
% report
This report considers methodology for apply Clever Algorithms for solving practical problems.

\begin{description}
	\item[Keywords:] {\small\texttt{Clever, Algorithms, Problem, Solving, Methodology}}
\end{description} 

% summarise the document breakdown with cross references
\section{Introduction}
\label{sec:introduction}
The Clever Algorithms project aims to describe a large number of algorithms from the fields of Computational Intelligence, Biologically Inspired Computation, and Metaheuristics in a complete, consistent and centralized manner \cite{Brownlee2010}.
% description
The project requires all algorithms to be described using a standardized template that includes a fixed number of sections, each of which is motivated by the presentation of specific information about the technique \cite{Brownlee2010a}.

% this report
The field of Data Mining has a clear methodologies that guides a practitioner to solve problems, such as Knowledge Discovery in Databases (KDD) \cite{Fayyad1996}. Metaheuristics and Computational Intelligence algorithms have no such methodology, although some methods can be used for classification and regression and as such may fit into methodologies such as KDD.

This report describes some of the considerations when applying algorithms from the fields of Metaheuristics, Computational Intelligence, and Biologically Inspired Computation to practical problem domains. This discussion includes:

\begin{itemize} 
  \item A discussion on the suitability of application of a given technique and the transferability of algorithm and problem features (Section~\cite{sec:suitability})
  \item other..
\end{itemize}

%
% Suitability of Application
%
\section{Suitability of Application}
\label{sec:suitability}
% utility as problems solvers
From a problem-solving perspective, the tools that emerge from the field of Computational Intelligence are generally assessed with regard to their utility as \emph{efficiently} or \emph{effectively} solving problems.
% nfl
An important lesson from the `no-free-lunch theorem' was to \emph{bound claims of applicability}. An approach toward this end is to consider the suitability of a given strategy with regard to the feature overlap with the attributes of a given problem domain. From a Computational Intelligence perspective, one may consider the architecture, processes, and constraints of a given strategy as the features of an approach. 

The suitability of the application of an `approach' to a `problem' takes into considerations concerns such as the \emph{appropriateness} (can the approach address the problem), the \emph{feasibility} (available resources and related efficiency concerns), and the \emph{flexibility} (ability to address unexpected or unintended effects).
% my approach
This section summarizes a general methodology toward addressing the problem of suitability in the context of Computational Intelligence tools. This methodology involves (1) the systematic elicitation of system and problem features, and (2) the consideration of the overlap of problem-problem, algorithm-algorithm, and problem-algorithm overlap of feature sets. 

% Systematic Feature Elicitation
\subsection{Systematic Feature Elicitation}
A \emph{feature} of a system (tool, strategy, model) or a problem is a distinctive element or property that may be used to differentiate it from similar and/or related cases. Examples may include functional concerns such as: processes, data structures, architectures, and constraints, as well as emergent concerns that may have a more subjective quality such as general behaviors, organizations, and higher-order structures. The process of the elicitation of features may be taken from a system or problem perspective as follows:

\begin{itemize}
	\item \emph{System Perspective}: This requires a strong focus on the lower level functional elements and investigations that work toward correlating specific controlled organizations towards predictable emergent behaviors. 
	\item \emph{Problem Perspective}: May require both a generalization of the specific case to the general problem case, as well as a functional or logical decomposition into constituent parts.
\end{itemize}

Problem \emph{generalization} and \emph{functional decomposition} are important and well used patterns for problem solving in the broader fields of Artificial Intelligence and Machine Learning as the promotion of simplification and modularity can reduce the cost and complexity of achieving solutions \cite{Russell2009, Brooks1986}.

%
% Feature Overlap
%
\subsection{Feature Overlap}
% general overlap
Overlap in elicited features may be considered from three important perspectives: \emph{between systems}, \emph{between problems}, and \emph{between a system and a problem}. Further, such overlap may be considered at different levels of detail with regard to generalized problem solving strategies and problem definitions.
% cases
These overlap cases are considered as follows:

\begin{itemize}
	\item \emph{System Overlap}: Defines the suitability of comparing one system to another, referred to as \emph{comparability}. For example, systems may be considered for the same general problems and compared in terms of theoretical or empirical capability, the results of which may only be meaningful if the systems are significantly similar to each other as assessed in terms of feature overlap. 
	\item \emph{Problem Overlap}: Defines the suitability of comparing one problem to another, referred to as \emph{transferability}. From a systems focus for example, transferability refers to the capability of a technique on a given problem to be transferred to another problem, the result of which is only meaningful if there is a strong overlap between the problems under consideration.
	\item \emph{System-Problem Overlap}: Defines the suitability of a system on a given problem, referred to as \emph{applicability}. For example, a system is considered suitable for a given problem if it has a significant overlap in capabilities with the requirements of the problem definition.
\end{itemize}

% noise
Such mappings are expected to have noise given the subjective assessment and/or complexity required in both the elicitation and consideration overlap of the of features, the noisiest of which is expected to be the mapping between systems and problems. 
% nfl
The mapping of salient features of algorithms and problems was proposed as an important reconciliation of the `no-free-lunch theorem' by Wolpert and Macready \cite{Wolpert1997}, although the important difference of this approach is that the system and algorithm are given prior to the assessment. In \cite{Wolpert1995}, Wolpert and Macready specifically propose the elicitation of the features from a problem first perspective, for which specialized algorithms can be defined. Therefore, this methodology of suitability may be considered a generalization of this reconciliation suitable for the altered `Computational Intelligence' (strategy first) perspective on Artificial Intelligence.
% this argument also supports outlining capability by analogy

%
% Strong and Weak Methods
%
\section{Strong and Weak Methods}
\label{sec:strong_methods}
% weak
Generally, the methods from the fields of Metaheuristics, Computational Intelligence, and Biologically Inspired Computation may be considered weak methods. They are weak because they are general purpose problem solves that are typically considers black-box solvers for a range of problem domains. The stronger the method, the more that must be known about the problem domain.
% weak and strong
Discriminating techniques into weak and strong is not a useful classification, it is better to consider a continuum of methods from pure block box techniques that have none or few assumptions about the problem domain, to strong methods that exploit most or all of the problem specific information available.

For example, the Traveling Salesman Problem is an example of a combinatorial optimization problem. A na\"ive (such a Random Search) black box method may simple explore permutations of the cities. Slightly stronger methods may initialize the search with a heuristic-generated technique (such as nearest neighbor) and explore the search space using a variation method that also exploits heuristic information about the domain (such as a 2-opt variation). Continuing along this theme, a stochastic method may explore the search space using a combination of probabilistic and heuristic information (such as Ant Colony Optimization algorithms). At the other end of the scale the stochastic elements are decreased or removed until one is left with  dynamic programming methods such as Lin-Kernighan and Concord.

% strategy
Approaching a problem is not as simple as selecting the strongest method available and solving the problem. The following describes two potential strategies:

\begin{itemize}
  \item \emph{Start Strong}: Select the strongest technique available and apply it to the problem. Difficult problems can be resistant to traditional methods for many intrinsic and extrinsic reasons. Use products from a strong technique (best solution found, heuristics) to seed the next weaker method in line. 
  \item \emph{Start Weak}: Strong methods do not exist for all problems, and if they do exist, the computation, skill, and/or time resources may not be available to exploit them. Start with a weak technique and use it to learn about the problem domain. Use this information to make decisions or better decisions about subsequent techniques to try that can exploit what has been learned.
\end{itemize}

In a real-world engineering or business scenario, the objective is to solve a problem, or achieve the best possible solution to the problem within the operating constraints.
Both of the above strategies suggest an iterative methodology, where the product or knowledge gained from one technique may be used to prime a subsequent stronger or weaker technique. 

%
% Local and Global Search
% 
\section{Local and Global Search}
As with the continuum between strong and weak methods for solving a problem, one may also consider methods that operate locally and globally in a search space. A global search method will consider the entire domain in an effort to locate an a solution, typically an approximate solution. A local search technique will focus on a constrained area of the search space.

Hill climbing algorithms are generally considered local search methods, and it is common for black-box stochastic optimization algorithms to provide a global search capability.
A general strategy when searching for a `good enough' solution to a difficult optimization problem is to start with a global search method and refine the result with a local search method. The application of the global search may be repeated to provide different potentially-good starting positions for the local search method (referred to as random restarts).

As with the weak-strong classification of methods, this local-global taxonomy of algorithms can also rapidly break down when specific algorithms are examined.
Some methods focus or converge on a area of the domain as they progress, ultimately arriving at a single solution (such as the Genetic Algorithm). Other methods can alternate between variations local and global search procedures under the banner of a single algorithm (such as GRASP). 



% summarise the document message and areas for future consideration
\section{Conclusions}
\label{sec:conclusions}
This report provided a discussion of some considerations when applying Metaheuristic and Computational Intelligence algorithms.
% source
Some of the content of this report was derived from the dissertation work of the author \cite{Brownlee2008}.

% bibliography
\bibliographystyle{plain}
\bibliography{../bibtex}

\end{document}
% EOF
